\documentclass{article}
\usepackage{graphicx} % Required for inserting images
\usepackage{booktabs}
\usepackage{amsmath}
\usepackage[a4paper, total={6.5in, 9.5in}]{geometry}
\usepackage{hyperref}
\usepackage{float}

\title{Previous Relationships vs Match Rate}
\author{Group 6}
\date{April 2025}

\begin{document}

\maketitle

\begin{table}[ht]
    \centering
    \begin{tabular}{ccc} 
    \toprule
    \textbf{First Name} & \textbf{Last Name} & \textbf{Student Number} \\ 
    \midrule
    Dean & Kuurstra    & s3343715 \\ 
    Diego & Cañas Jimenez & s3856216 \\
    Koorosh & Komeili Zadeh & s3893995 \\ 
    Lani & Hampel & s3977412 \\ 
    \bottomrule
    \end{tabular}
    \label{tab:group6_members}
\end{table}

\section{Introduction}

According to Stanford University Libraries (2023) \cite{rosenfeld2023}, the most common way couples meet is through online dating. This makes online dating a relevant topic of study and interest. To further understand online dating, it is worth considering distinct variables and their effect on the number of matches a user receives. This paper will investigate how the number of previous relationships of a user influences their match rates. This may provide further insight into the most common method currently used to find relationships. Furthermore, research into this topic may uncover unknown trends with less obvious features, one of which could be the aforementioned number of previous relationships and a user’s likelihood of successfully finding a match through online dating.

\section{Exploration}

% Simplify and shorten the paragraph

It is interesting to explore the relationship between the number of previous relationships and match rate, as reasonable arguments can defend both (positive and negative) trends. On the one hand, individuals with a history of relationships may have developed better interpersonal skills, emotional intelligence, and self-awareness, which could make them more attractive to potential matches. Additionally, there is a fundamental instinct to desire what others like, which could make individuals who have had more dates seem more appealing. Conversely, having many previous relationships could also indicate commitment issues or an inability to maintain long-term relationships, which might make users less appealing to potential matches. There could also exist the possibility that both variables are not related; whether experience in relationships is an advantage or a liability is an open question that could provide meaningful insights into online dating if answered.



\section{Methodology}

Our study aims to uncover the relationship between the number of previous relationships and match rate. For this relationship, we use the number of previous relationships as an explanatory variable, and we use the match rate as the response variable. We aim to examine the relationship through linear regression and thus answer whether the previous number of relationships affects the match rate of the users, whether this effect is positive or negative, if it is indeed present, and how strong this relationship is. 

\section{Hypothesis}

Our Alternative Hypothesis points to the relationship between the number of previous relationships of a user and their match rate, and the Null Hypothesis argues that there is no direct connection between the number of previous significant relationships of a user and said user's match rate.

\section{References}
\bibliographystyle{IEEEtran}
\bibliography{references}

\end{document}
